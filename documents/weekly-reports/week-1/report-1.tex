\documentclass[a4paper,11pt]{article}
\usepackage[T1]{fontenc}
\usepackage[utf8]{inputenc}
\usepackage{lmodern}
\usepackage{listings}

\title{Report 1}

\begin{document}

\section{Weekly Meeting(s):}
Meeting time: 6:00 pm - 9:30 pm (3 1/2 hours)\\
All team members present

\section{Deliverables achieved:}
\begin{enumerate}
\item	onboarding: setup repos, Git, IDE (intelliJ/VIM)
\item determined our Github workflow
\item reviewed the rules of the game and played a mock round
\item decided that we are going to use Java and JavaFX
\item set up our re-occuring out-of-class meeting (Thursdays 6:00pm-8:00pm)
\end{enumerate}

\section{Action Items due Tuesday January 23rd:}
\begin{enumerate}
\item type and share Github workflow (Carolyne)
\item look into purchasing game (Carolyne)
\item post questions on CU Learn (Akhil)
\item finalize progress report (together) - see progress report information below from prof
\item Read rules (Carolyne)
\item Weekly Report Template (Carolyne)
\item Set up Trello (Carolyne)
\item Meeting Sunday at noon - max 1

\end{enumerate}

\section{Scenario:}
\paragraph{}
Your first progress report is due Tuesday the 23rd (see its details under Topic 1)
In this first one, we want to specifically see a set of scenarios that you develop to test the game.
We are NOT imposing any particular syntax to use for you to describe this scenarios.
We want as few scenarios as possible that test as many of the features you plan to implement in this first iteration.
Think of a scenario as the description of a game:
\paragraph{}

\begin{lstlisting}
For example:
scenario 1:
game is set up for 3 players
player 1 receives <some list of cards>
player 2 receives <some list of cards>
player 3 receives <some list of cards>
quest "Boar Hunt" is drawn by player 1 who accepts setting it up
stage 1: <some list of cards> stage 2: <some list of cards>
player 2 is asked if she participates and declines
player 3 is asked if he participates and accepts
player 3 gets one adventure card and stage 1 is resolved: player 3 does 
	not pass stage 1
player 1 discards cards used for the quest and replaces them
-etc.	
\end{lstlisting}

\paragraph{}
That is, before coding, decide on the features you are going for and write a corresponding test plan (which will evolve).
\paragraph{}
A feature is a SMALL piece of functionality: it must be codable and tested in hours or very few days, not weeks.
\paragraph{}
That means, you MUST break big requirements into small features...
\paragraph{}
Rank your features in decreasing order of importance (as it is likely you will not get to the latter ones...)
\paragraph{}
Then decide on the language and tools and set up your repository. (DONE)
\paragraph{}
Then do NOT distribute all features among team members.
\paragraph{}
Instead, starting with the most important ones, tackle them so that you DO complete at least one by the time your progress report is due.
\paragraph{}
Remember: the ability to 'rig' your game (by assigning specific cards to specific players and setting up the decks) is crucial for the testability of your project. This is an implicit requirement since it is not something a customer will ask for BUT it is something you must provide in order to demonstrate to the customer that you have satisfied his requirements.

\section{Scenarios}

\subsection{Basic Quest}
meta-data:\\
Number of Players : 2\\
Number of cards Dealt: 12\\
Initial frequency : 4 weapons, 5 foes, 1 armor, 1 ally, 1 test\\
\begin{enumerate}
	\item P1 draws Vanguerth Quest (3 stages)
	\item P1 sponsors (stage 1 : thieves +5; stage 2 : test of Valor; stage 3: Boar + 5, horse +10)
	\item P2 participates in quest (13 cards)
	\item P2 plays Sir Galahad to make hand equal to 12
	\item P2 turn : plays dagger; total BP =  5 rank + 15 Ally + 5 weapon = 25
	\item P2 Passes stage 25 > 5
	\item P2 draws card (12 cards in hand)
	\item Test of Valor Revealed 
	\item P2 bids 3 cards from hand (9 total in hand)
	\item P2 passes stage 2 : Test and draws 1 ( 10 total in hand)
	\item P2 turn : plays horse, and amour; total BP = 10 + 10 + 15 + 5 = 50 > 15
	\item P2 passes stage 3
	\item P2 awarded 3 shields
	\item P1 draws 7 cards and discards accordingly to  maximum of 12 rule
	\item P2 draws from story deck
\end{enumerate}



\subsection{Event : Court Call to Camelot}
meta-data:\\
P1 : shields = 0, cards in play: Sir Galahad\\
P2 : shields = 3\\
2 players\\
\begin{enumerate}
	\item P2 draws an event : Court Call to Camelot
	\item P1 does nothing
	\item P2 discards Sir Galahad
\end{enumerate}

\subsection{Event : Pox}
meta-data:\\
P1 : shields = 0, cards in play: Sir Galahad\\
P2 : shields = 3\\
2 players\\
\begin{enumerate}
	\item P2 draws an event : Pox
	\item P1 losses 1 shield, but no shields are lost since shields are 0
	\item P2 does nothing
\end{enumerate}


\subsection{Event : Pox}
meta-data:\\
P1 : shields = 0, cards in play: Sir Galahad\\
P2 : shields = 3\\
2 players\\
\begin{enumerate}
	\item P1 draws an event : Pox
	\item P2 losses 1 shield (total shields : 2)
	\item P1 does nothing
\end{enumerate}


\subsection{Event : Plague}
meta-data:\\
P1 : shields = 0, cards in play: Sir Galahad\\
P2 : shields = 3\\
2 players\\
\begin{enumerate}
	\item P2 draws an event : Plague 
	\item P1 does nothing
	\item P2 losses 2 shields (total shields : 1)
\end{enumerate}


\subsection{Event : Chivalrous Deed}
meta-data:\\
P1 : shields = 0, cards in play: Sir Galahad, rank : squire\\
P2 : shields = 3, rank : squire \\
2 players\\
\begin{enumerate}
	\item P2 draws an event : Chivalrous Deed
	\item P1 receives 3 shields total (total shields : 3)
	\item P2 does nothing
\end{enumerate}


\subsection{Event : Chivalrous Deed}
meta-data:\\
P1 : shields = 0, cards in play: Sir Galahad, rank : squire\\
P2 : shields = 3, rank : squire\\
P3 : shields = 0, rank : squire\\
2 players\\
\begin{enumerate}
	\item P2 draws an event : Chivalrous Deed
	\item P1 receives 3 shields total (total shields : 3)
	\item P3 receives 3 shields total (total shields : 3)
	\item P2 does nothing
\end{enumerate}


\subsection{Event : Prosperity}
meta-data:\\
P1 : shields = 0, cards in play: Sir Galahad\\
P2 : shields = 3\\
2 players\\
\begin{enumerate}
	\item P2 draws an event : Prosperity 
	\item P1 draws 2 cards
	\item P2 draws 2 cards
	\item discard is handled accordingly 
\end{enumerate}

\subsection{Event : Kings Call to Arms}
meta-data:\\
P1 : shields = 0, cards in play: Sir Galahad, rank : squire\\
P2 : shields = 3, rank : knight\\
2 players\\
\begin{enumerate}
	\item P2 draws an event : Kings Call to arms 
	\item P1 does nothing
	\item P2 discards a weapon 
\end{enumerate}
\end{document}
